\documentclass[draft,10pt,oneside]{scrartcl}

% Sprache und Encodings
\usepackage[ngerman]{babel}
%\usepackage[T1]{fontenc}
\usepackage[utf8]{inputenc}

% Typographisch interessante Pakete
\usepackage{microtype} % Randkorrektur und andere Anpassungen

% References to Internet and within the document
\usepackage[pdftex,colorlinks=false, pdftitle={Antrag zur Änderung der
Geschäftsordnung für Plenen der ZaPF}, pdfauthor={Jörg Behrmann (FUB), Björn
Guth (RWTH)}, pdfcreator={pdflatex}, pdfdisplaydoctitle=true]{hyperref}

% Absaetze nicht Einruecken
\setlength{\parindent}{0pt} \setlength{\parskip}{2pt}

% Formatierung auf A4 anpassen
\usepackage{geometry}
\geometry{paper=a4paper,left=15mm,right=15mm,top=10mm,bottom=10mm}

% Zeilenumbrüche
\usepackage{hyphenat}
%\hyphenation{Baden-Württemberg}

\begin{document}

\section*{}

\textbf{Antragsteller:} Jörg Behrmann (FUB), Björn Guth (RWTH)

\textbf{Adressaten:} alle Fraktionen aller deutschen Landtage

\subsection*{Antrag}

Die ZaPF möge beschließen:

\begin{quote}
Die ZaPF widerspricht allen Bestrebungen zum Rückbau universitärer Demokratie
und einer Rückkehr zu den Zuständen der Ordinarienuniversität, wie sie vor 1968
existierte.

Die Universität ist nicht nur der Arbeitsplatz von Professorika, sondern ein Ort
an dem viele verschiedene Menschen lehren, lernen und arbeiten. Die idealen
Bedingungen dafür können nur durch Teilhabe und die Vertretung der Interessen
aller her- und sichergestellt werden. Dafür müssen alle Statusgruppen angemessen
in allen Bereichen, insbesondere allen relevanten Räten und Senaten, vertreten
sein.

Da die Universität ein Abbild der gesamten Gesellschaft darstellen sollte, muss
auch benachteiligten Gruppen der Gesellschaft der Zugang zu Universitäten und
universitärer Bildung ermöglicht werden. Frauen- und
Gleichstellungsbeauftragtika haben sich dafúr als bewährtes Mittel
erwiesen. Ihre Teilnahme an oben genannten universitären Gremien ist daher
unerlässlich.
\end{quote}

\subsection*{Begründung}

Der Auslöser dieses Antrages ist Drucksache 7/3844 des Landtages von Sachsen
Anhalt
(\url{https://www.landtag.sachsen-anhalt.de/fileadmin/files/drs/wp7/drs/d3844aan.pdf})
in dem die sogenannte "AfD" einem Universitätsbild aus dem 50er-Jahren
hinterhertrauert. Wir müssen uns dagegen verwahren die Zeit dorthin
zurückzudrehen.

\end{document}

%%% Local Variables:
%%% mode: latex
%%% TeX-master: t
%%% End:
