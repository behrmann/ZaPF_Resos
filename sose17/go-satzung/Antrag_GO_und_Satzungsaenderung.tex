\documentclass[draft,10pt,oneside]{scrartcl}

% Sprache und Encodings
\usepackage[ngerman]{babel}
\usepackage[T1]{fontenc}
\usepackage[utf8]{inputenc}

% Typographisch interessante Pakete
\usepackage{microtype} % Randkorrektur und andere Anpassungen

% References to Internet and within the document
\usepackage[pdftex,colorlinks=false,
pdftitle={Antrag zur Änderung der Geschäftsordnung für Plenen und der Satzung der ZaPF},
pdfauthor={Jörg Behrmann (FUB), Björn Guth (RWTH)},
pdfcreator={pdflatex},
pdfdisplaydoctitle=true]{hyperref}

% Absaetze nicht Einruecken
\setlength{\parindent}{0pt}
\setlength{\parskip}{2pt}

% Formatierung auf A4 anpassen
\usepackage{geometry}
\geometry{paper=a4paper,left=15mm,right=15mm,top=10mm,bottom=10mm}

\begin{document}

\section*{Antrag zur Änderung der Geschäftsordnung für Plenen der ZaPF}

\textbf{Antragsteller:} Jörg Behrmann (FUB), Björn Guth (RWTH)

\subsection*{Antrag}

Hiermit beantragen wir die Geschäftsordnung für Plenen der ZaPF wie folgt zu
ändern:

Absatz 4.2.9
\begin{quote} Abwahlen sind auch bei Abwesenheit der betroffenen Person möglich
und bedürfen einer Zweidrittelmehrheit. \emph{Die betroffene Person ist jedoch
nach Möglichkeit anzuhören.}
\end{quote} ist zu ändern in
\begin{quote} Abwahlen sind auch bei Abwesenheit der betroffenen Person möglich
und bedürfen einer Zweidrittelmehrheit. \emph{Der Antrag auf Abwahl ist bis
spätestens 15 Uhr am Vortag der ausrichtenden Fachschaft anzukündigen. Die
betroffene Person ist jedoch nach Möglichkeit anzuhören.}
\end{quote}

\subsection*{Begründung}

Im Arbeitskreis wurde diskutiert ob die Modi von Abwahlen sauber aus der Satzung
getrennt werden sollen, da alle Wahlmodi in der GO geregelt sind. Dies wurde aus
einem besonderen Schutzinteresse dieser Regelungen abgelehnt, um aber alle
Regelungen an einem Ort zu vereinen und so das Verständnis zu erleichtern,
sollen sie hier bewusst gedoppelt werden.

\newpage

\section*{Antrag zur Änderung der Geschäftsordnung für Plenen der ZaPF}

\textbf{Antragsteller:} Jörg Behrmann (FUB), Björn Guth (RWTH)

\subsection*{Antrag}

Hiermit beantragen wir die Geschäftsordnung für Plenen der ZaPF wie folgt zu
ändern:

In 4.2.6 ersetze
\begin{quote}
  Im Anfangsplenum werden sechs Vertrauenspersonen gewählt. Zur Wahl berechtigt
  sind alle \emph{angemeldeten Teilnehmer der ZaPF}.
\end{quote}
durch
\begin{quote}
  Im Anfangsplenum werden sechs Vertrauenspersonen gewählt. Zur Wahl berechtigt
  sind alle \emph{im Plenum anwesenden Personen}.
\end{quote}
und 4.2.7
\begin{quote}
  Die Wahl der Vertrauenspersonen erfolgt per Wahl durch Zustimmung aus einem
  Pool von \emph{angemeldeten Teilnehmern} der ZaPF
\end{quote}
durch
\begin{quote}
  Die Wahl der Vertrauenspersonen erfolgt per Wahl durch Zustimmung aus einem
  Pool von \emph{teilnehmende Personen} der ZaPF
\end{quote}

\subsection*{Begründung}

Die ursprüngliche Intention der obigen Regelungen war jeder im Plenum anwesenden
Person eine für sie akzeptable Vertrauensperson zu finden. Dies wird durch diese
Formulierung ermöglicht, da sie sich nicht nur auf teilnehmende bzw angemeldete
Personen beschränkt.

\newpage


\end{document}

%%% Local Variables:
%%% mode: latex
%%% TeX-master: t
%%% End:
