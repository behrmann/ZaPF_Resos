\documentclass[draft,10pt,oneside]{scrartcl}

% Sprache und Encodings
\usepackage[ngerman]{babel}
\usepackage[T1]{fontenc}
\usepackage[utf8]{inputenc}

% Typographisch interessante Pakete
\usepackage{microtype} % Randkorrektur und andere Anpassungen

% References to Internet and within the document
\usepackage[pdftex,colorlinks=false,
pdftitle={Antrag zur Änderung der Geschäftsordnung für Plenen und der Satzung der ZaPF},
pdfauthor={Jörg Behrmann (FUB), Björn Guth (RWTH)},
pdfcreator={pdflatex},
pdfdisplaydoctitle=true]{hyperref}

% Absaetze nicht Einruecken
\setlength{\parindent}{0pt}
\setlength{\parskip}{2pt}

% Formatierung auf A4 anpassen
\usepackage{geometry}
\geometry{paper=a4paper,left=15mm,right=15mm,top=10mm,bottom=10mm}

\begin{document}

\section*{Antrag zur Änderung der Geschäftsordnung für Plenen der ZaPF}

\textbf{Antragsteller:} Jörg Behrmann (FUB), Björn Guth (RWTH)

\subsection*{Antrag}

Hiermit beantragen wir die Geschäftsordnung für Plenen der ZaPF wie folgt zu
ändern:

Absatz 4.2.9
\begin{quote} Abwahlen sind auch bei Abwesenheit der betroffenen Person möglich
und bedürfen einer Zweidrittelmehrheit. \emph{Die betroffene Person ist jedoch
nach Möglichkeit anzuhören.}
\end{quote} ist zu ändern in
\begin{quote} Abwahlen sind auch bei Abwesenheit der betroffenen Person möglich
und bedürfen einer Zweidrittelmehrheit. \emph{Der Antrag auf Abwahl ist bis
spätestens 15 Uhr am Vortag der ausrichtenden Fachschaft anzukündigen. Die
betroffene Person ist jedoch nach Möglichkeit anzuhören.}
\end{quote}

\subsection*{Begründung}

Im Arbeitskreis wurde diskutiert ob die Modi von Abwahlen sauber aus der Satzung
getrennt werden sollen, da alle Wahlmodi in der GO geregelt sind. Dies wurde aus
einem besonderen Schutzinteresse dieser Regelungen abgelehnt, um aber alle
Regelungen an einem Ort zu vereinen und so das Verständnis zu erleichtern,
sollen sie hier bewusst gedoppelt werden.

\newpage

\section*{Antrag zur Änderung der Geschäftsordnung für Plenen der ZaPF}

\textbf{Antragsteller:} Jörg Behrmann (FUB), Björn Guth (RWTH)

\subsection*{Antrag}

Hiermit beantragen wir die Geschäftsordnung für Plenen der ZaPF wie folgt zu
ändern:

In 4.2.6 ersetze
\begin{quote}
  Im Anfangsplenum werden sechs Vertrauenspersonen gewählt. Zur Wahl berechtigt
  sind alle \emph{angemeldeten Teilnehmer der ZaPF}.
\end{quote}
durch
\begin{quote}
  Im Anfangsplenum werden sechs Vertrauenspersonen gewählt. Zur Wahl berechtigt
  sind alle \emph{anwesenden natürlichen Personen}.
\end{quote}
und 4.2.7
\begin{quote}
  Die Wahl der Vertrauenspersonen erfolgt per Wahl durch Zustimmung aus einem
  Pool von \emph{angemeldeten Teilnehmern} der ZaPF
\end{quote}
durch
\begin{quote}
  Die Wahl der Vertrauenspersonen erfolgt per Wahl durch Zustimmung aus einem
  Pool von \emph{teilnehmenden Personen} der ZaPF
\end{quote}

\subsection*{Begründung}

Die ursprüngliche Intention der obigen Regelungen war jeder im Plenum anwesenden
Person eine für sie akzeptable Vertrauensperson zu finden. Dies wird durch diese
Formulierung ermöglicht, da sie sich nicht nur auf teilnehmende bzw.\  angemeldete
Personen beschränkt.

\newpage

\section*{Antrag zur Änderung der Geschäftsordnung für Plenen der ZaPF}

\textbf{Antragsteller:} Jörg Behrmann (FUB), Björn Guth (RWTH)

\subsection*{Antrag}

Hiermit beantragen wir die Geschäftsordnung für Plenen der ZaPF wie folgt zu
ändern:

Abschnitt 3.1 Antragsfristen und Antragsdurchführung ist ein neuer Artikel 5
\begin{quote}
  Anträge, die bestehende Aussagen der ZaPF, insbesondere die Geschäftsordnung
  und die Satzung, ändern wollen, müssen ihre Änderung des bestehenden Textes
  geeignet nachvollziehbar machen.

  Diese Pflicht entfällt für Initiativanträge.
\end{quote}
und dem Anhang ein Abschnitt
\begin{quote}
  \textbf{Geeignete Form des Nachvollziehbarmachens}

  Es kann sehr schwer sein kleinste Änderungen in Texten nachzuvollziehen, es
  erleichtert die Arbeit im Plenum deswegen erheblich, wenn Änderungen
  bestehender Texte im einzelnen hervorgehoben sind. Dies kann z.B. durch ein
  Diff geschehen.
\end{quote}

\subsection*{Begründung}

Die Begründung ist dem obigen neuen Anhang zu entnehmen.

\newpage

\section*{Antrag zur Änderung der Geschäftsordnung für Plenen der ZaPF}

\textbf{Antragsteller:} Jörg Behrmann (FUB), Björn Guth (RWTH)

\subsection*{Antrag}

Hiermit beantragen wir die Geschäftsordnung für Plenen der ZaPF wie folgt zu
ändern:

In 4.1.2 sind die Wörter ``Resolution'', ``Positionspapier'' und
``Selbstverpflichtung'' hervorzuheben um auf den neuen Anhang
\begin{quote}
  \textbf{Resolutionen, Positionspapiere und Selbstverpflichtungen}

  Resolutionen halten Positionen der ZaPF fest und werden vom StAPF an die im
  Antrag angegebenen Adressaten verschickt.

  Positionspapiere erfüllen den selben Zweck wie Resolutionen, aber haben keine
  eigenen Adressaten und sollen nur im Bericht des StAPFes und auf der
  Internetpräsenz der ZaPF in der Liste aller Resolutionen und Positionspapiere
  veröffentlicht werden.

  Selbstverpflichtungen sind ZaPF-interne Dokumente, die Aufträge an die Organe
  der ZaPF, z.B. den StAPF, geben. Selbstverpflichtungen können insbesondere
  dafür genutzt werden Arbeitsthesen eines Arbeitskreises festzuhalten, mit der
  Intention auf einer folgenden ZaPF einen weiteren Beschluss zu fassen.
\end{quote}
hinzuzuweisen, der der Geschäftsordnung hinzuzufügen ist.

\subsection*{Begründung}

Die Erfahrungen des Workshops ``Resoschreiben'' und vergangener Plena haben
gezeigt, dass vielen Teilnehmika der Unterschied zwischen Resolutionen,
Positionspapieren und Selbstverpflichtungen nicht klar ist.

Weiterhin wurde im aktuellen Workshop ``Resoschreiben'' ausgiebig diskutiert ob
weitere Klassen von Anträgen nicht notwendig sind um den Zwischenstand einer
Arbeit für nachfolgende Arbeiten zu sichern, aber in einer Art, die nicht
ZaPF-extern einsehbar ist.

Eine Notwendigkeit für neue Klassen von Anträgen wurde im Workshop abschließend
verneint, aber der obige erklärende Anhang formuliert.

\newpage

\section*{Antrag zur Änderung der Geschäftsordnung für Plenen der ZaPF}

\textbf{Antragsteller:} Jörg Behrmann (FUB), Björn Guth (RWTH)

\subsection*{Antrag}

Hiermit beantragen wir die Geschäftsordnung für Plenen der ZaPF wie folgt zu
ändern:

Der GO-Antrag auf Nichtbefassung in der Liste der GO-Anträge ist wie folgt zu
ändern:
\begin{quote}
  \begin{itemize}
  \item Nichtbefassung *
  \end{itemize}
\end{quote}
zu
\begin{quote}
  \begin{itemize}
  \item Nichtbefassung \emph{(kann nicht geheim abgestimmt werden)} *
  \end{itemize}
\end{quote}

\subsection*{Begründung}

Der ebenfalls vorliegende Satzungsänderungsantrag zur genaueren Spezifizierung
der Aufgaben der ZaPF soll mit dieser GO-Änderung vorbereitet werden.

Gegen die genannte Satzungsänderung, die die bestehende Politik der ZaPF
erklärt, sich zu allgemeinpolitischen Themen zu äußern, die einen klaren
Hochschulbezug haben, wurde vorgebracht, dass vereinzelte Fachschaften eine
engere Vorgabe haben könnten, die ihnen auch solche Äußerungen verbietet. Diese
können durch bestehende Mittel der Geschäftsordnung beweisen, dass sie sich der
Befassung eines solchen Themas enthalten haben.

Ein solches Mittel wäre der GO-Antrag auf Nichtbefassung in namentlicher
Abstimmung. Da geheime Abstimmung namentliche Abstimmung um allgemeinen schlägt,
soll dieses für diesen GO-Antrag explizit ausgeschlossen werden.

\newpage

\section*{Antrag zur Änderung der Satzung der ZaPF}

\textbf{Antragsteller:} Jörg Behrmann (FUB), Björn Guth (RWTH)

\subsection*{Antrag}

Hiermit beantragen wir die Satzung der ZaPF wie folgt zu ändern:

Der bestehende Artikel 3 ``Aufgaben''
\begin{quote}
Die ZaPF findet einmal pro Semester statt; sie tagt öffentlich. Sie befasst
sich mit hochschul- und studienrelevanten Themenbereichen.

Die ZaPF dient dem Sammeln und der Diskussion von Informationen zu diesen Themen
und tritt mit Resultaten gegebenenfalls an die Öffentlichkeit, besitzt aber kein
allgemeinpolitisches Mandat.
Des Weiteren dient sie zum Gedanken- und Ideenaustausch zwischen den Fachschaften.
\end{quote}
ist zu ersetzen durch
\begin{quote}
Die ZaPF findet einmal pro Semester statt und tagt öffentlich. Sie dient dem
Sammeln und der Diskussion von Informationen und tritt mit den Resultaten
gegebenenfalls an die Öffentlichkeit oder an Dritte heran.
Des Weiteren dient sie zum Gedanken- und Ideenaustausch zwischen den
Fachschaften.

Die ZaPF befasst sich mit studien- und hochschulrelevanten Themen. Sie besitzt
kein allgemeinpolitisches Mandat, kann sich jedoch in Bezug auf
hochschulpolitische Themen auch allgemeinpolitisch äußern. Hierbei muss ein
Zusammenhang zu studien- und hochschulpolitischen Belangen unmittelbar bestehen
und deutlich erkennbar bleiben.
\end{quote}
und ein Anhang
\begin{quote}
\emph{Politisches Mandat}

Die Fachschaften als Teil der Verfassten Studierendenschaften haben nach
gängiger Rechtsauffassung kein allgemeinpolitisches Mandat. Es ist ihnen deshalb
verboten allgemeinpolitische Meinungen und Forderungen zu formulieren und zu
propagieren. Zudem dürfen sie auch Dritte, die ein allgemeinpolitisches Mandat
beanspruchen und entsprechende Aktivitäten entfalten nicht durch Mitarbeit,
Geld- oder Sachzuwendungen unterstützen.

Hierbei ist es irrelevant, ob sich die einzelnen Fachschaften eine Satzung
gegeben haben, welche ein allgemeinpolitisches Mandat ausschließt, oder nicht.

Allerdings räumte das Bundesverfassungsgericht mit seinem Urteil vom 4. August
2000 [BVerfG Az. 1 BvR 1510/99] den Studierendenschaften (und damit den
Fachschaften) die Möglichkeit eines sogenannten Brückenschlags ein, wonach bei
der Behandlung hochschulpolitischer Themen allerdings ein ``Brückenschlag'' zu
allgemeinpolitischen Fragestellungen erlaubt ist, solange und soweit dabei der
Zusammenhang zu studien- und hochschulpolitischen Belangen unmittelbar besteht
und deutlich erkennbar bleibt.

Zahlreiche weitere Urteile von Gerichten stecken hierbei den Rahmen mehr und
mehr ab. Beispielurteile:

\begin{itemize}
\item Studentenschaft Universität Münster (2. Oktober 1996, OVG Münster)
\item Studentenschaft Universität Bonn (1996, VG Köln, 6 L 28/96)
\item Studentenschaft Universität Wuppertal (1996, VG Düsseldorf, 15 L 781/96)
\item Studentenschaft Freie Universität Berlin (Oberverwaltungsgericht
  Berlin, 15. Januar 2004, 8 S 133.03)
\item Studentenschaft Universität Trier (Oberverwaltungsgericht Koblenz,
  Beschluss vom 28. Januar 2005 - 2 B 12002/04)
\item Studentenschaft Humboldt-Universität Berlin (Oberverwaltungsgericht
  Berlin, Beschluss vom 4. Mai 2005 - 8 N 196.02)
\end{itemize}

Es ist deshalb unerlässlich für die ZaPF den Anspruch an ein
allgemeinpolitisches Mandat abzulehnen. Allerdings kann sie unter den
erläuterten Umständen mit den gegebenen Mittel zu allgemeinpolitischen Themen
Meinungen und Forderungen bilden.

Haben einzelne Fachschaften ein enger gefasstes politisches Mandat, so können
diese einen GO-Antrag auf Nichtbefassung in Zusammenhang mit namentlicher
Abstimmung stellen um kenntlich zu machen, dass sie diesen Tagesordnungspunkt
nicht behandelt haben.
\end{quote}
ist der Satzung hinzuzufügen.

\subsubsection*{Begründung}

Diese Satzungsänderung erklärt das politische Mandat der ZaPF und spiegelt den
bisherigen Status Quo wider und bettet die Erklärung in den bestehenden
rechtlichen Rahmen ein. Dies macht die Satzung klarer und sollte in Zukunft
verhindern vereinzelte Diskussionen wieder und wieder zu führen.

\newpage

\section*{Antrag zur Änderung der Satzung der ZaPF}

\textbf{Antragsteller:} Jörg Behrmann (FUB), Björn Guth (RWTH)

\subsection*{Antrag}

Hiermit beantragen wir die Satzung der ZaPF wie folgt zu ändern:

Artikel 5(e) ``Der Technische Organisationsausschuss aller Physikfachschaften
(TOPF)'' ist der Absatz
\begin{quote}
  Die Amtszeit der Hauptverantwortlichen beträgt ein Jahr.
\end{quote}
hinzuzufügen.

\subsection*{Begründung}

Beim Einführen dieses Organs der ZaPF wurde vergessen die angedachte Amtszeit in
die Satzung zu schreiben.

\end{document}

%%% Local Variables:
%%% mode: latex
%%% TeX-master: t
%%% End:
